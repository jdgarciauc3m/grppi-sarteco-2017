\subsection{Programación Paralela}

\begin{frame}
\begin{LARGE}
\begin{center}
Thinking \onslide<1>{in Parallel}\onslide<2>{} is hard.
\end{center}
\end{LARGE}
\hfill \onslide<2>{Yale Patt\\}
\hfill \onslide<2>{CEDI 2007, Zaragoza}
\end{frame}

\begin{frame}[t]{Programación secuencial y programación paralela}
\begin{itemize}
\item \textmark{Programación secuencial}
  \begin{itemize}
    \item Conjunto bien conocido de \emph{estructuras de control} integradas en los lenguajes de programación.
    \item Estructuras de control inherentemente secuenciales.
  \end{itemize}
\vfill\pause
\item \textmark{Programación paralela}
  \begin{itemize}
    \item Construcciones que \emph{adaptan} las estructuras secuenciales al mundo paralelo
          (p.ej. \emph{parallel-for}).
  \end{itemize}
\vfill\pause
\item \textgood{Pero\ldots}
  \begin{itemize}
    \item Y si tuviesemos construcciones que pudiesen ser a la vez secuenciales y paralelas.
  \end{itemize}
\end{itemize}
\end{frame}
