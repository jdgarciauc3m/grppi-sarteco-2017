\subsection{Arquitectura de GrPPI}

\begin{frame}[t]{Ideales}
\begin{itemize}[<+->]
  \item Las aplicaciones deberían poder expresarse independientemente
        de su modelo de ejecución.
  \item Se debería poder usar múltiples \emph{back-ends} con mecanismos
        sencillos para cambiar de uno a otro.
  \item La interfaz de programación debería poderse integrar \emph{suavemente}
        con C++ moderno y su biblioteca estándar.
  \item Se debería poder aprovechar las ventajas de C++ moderno (C++14 y más allá).
\end{itemize}
\end{frame}

\begin{frame}[t]{GrPPI}
\begin{Large}
\textgood{\url{https://github.com/arcosuc3m/grppi}}
\end{Large}
\vfill\pause
\begin{itemize}
  \item Una biblioteca \textmark{header only}.
  \item Un conjunto de políticas de ejecución intercambiables.
  \item Un conjunto de algoritmos genéricos con seguridad de tipos.
  \item Se base en técnicas de metaprogramación sin exponerlas a los usuarios.
  \item Requiere \textgood{C++14}.
  \item GNU GPL v3.
\end{itemize}
\end{frame}

\begin{frame}[t]{Políticas de ejecución}
\begin{itemize}
  \item Encapsulan el modelo de ejecución.
  \vfill
  \item Políticas de ejecución disponibles:
    \begin{itemize}
      \item \cppid{sequential\_execution}.
      \item \cppid{parallel\_execution\_native}.
      \item \cppid{parallel\_execution\_omp}.
      \item \cppid{parallel\_execution\_tbb}.
      \item \cppid{parallel\_execution\_thrust} (pendiente de liberar).
      \item \cppid{dynamic\_execution}.
    \end{itemize}
  \vfill
  \item Todos los patrones toman un objecto \emph{ejecución}
  \vfill
  \item Una política de ejecución tiene propiedades asociadas.
    \begin{itemize}
      \item Ejemplo: \cppid{concurrency\_degree}.
    \end{itemize}
\end{itemize}
\end{frame}
