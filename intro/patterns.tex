\subsection{Patrones de diseño y patrones de paralelismo}

\begin{frame}[t]{Diseño de software}

\begin{quote}
There are two ways of constructing a software design: One way is to make it so
simple that there are obviously no deficiencies, and the other way is to make
it so complicated that there are no obvious deficiencies. 

The first method is far more difficult. 
\end{quote}
\hfill C.A.R Hoare
\end{frame}

\begin{frame}[t]{Una breve historia de los patrones}
\begin{itemize}
\item Desde la \textmark{construcción de edificios y ciudades} (Cristopher Alexander):
\begin{itemize}
  \item \textbf{1977}: A Pattern Language: Towns, Buildings, Construction.
  \item \textbf{1979}: The timeless way of buildings.
\end{itemize}
\vfill\pause
\item Pasando por el \textmark{diseño de software} (Gamma et al.):
\begin{itemize}
  \item \textbf{1993}: Design Patterns: abstraction and reuse of object oriented design. ECOOP.
  \item \textbf{1995}: Design Patterns. Elements of Reusable Object-Oriented Software.
\end{itemize}
\vfill\pause
\item Hasta llegar a la \textgood{programación paralela} (McCool, Reinders, Robinson):
\begin{itemize}
  \item \textbf{2012}: Structured Parallel Programming: Patterns for Efficient Computation.
\end{itemize}
\end{itemize}
\end{frame}
