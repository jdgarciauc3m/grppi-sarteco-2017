\begin{frame}[t]{Pipeline}
\begin{itemize}
  \item Un \textgood{pipeline} 
        permite procesar flujos de información cuando el cómputo se puede dividir en etapas.
    \begin{itemize}
      \item Cada etapa proces los datos generados por la etapa previa y pasa el resultado
            a la siguiente.
    \end{itemize}
\end{itemize}
\end{frame}

\begin{frame}[t,fragile]{Pipeline simple}
\begin{block}{x -> x*x -> 1/x -> print}
\begin{lstlisting}
template <typename Execution>
void run_pipe(const Execution & ex, int n)
{
  grppi::pipeline(ex,
    [i=0,max=n] () mutable -> optional<int> {
      if (i<max) return i;
      else return {};
    },
    [](int x) -> double { return x*x; },
    [](double x) { return 1/x; },
    [](double x) { cout << x << "\n"; }
  );
}
\end{lstlisting}
\end{block}
\end{frame}

\begin{frame}[t,fragile]{Pipelines anidados}
\begin{block}{x -> x*x -> 1/x -> print}
\begin{lstlisting}
template <typename Execution>
void run_pipe(const Execution & ex, int n)
{
  grppi::pipeline(ex,
    [i=0,max=n] () mutable -> optional<int> {
      if (i<max) return i;
      else return {};
    },
    grppi::pipeline(
      [](int x) -> double { return x*x; },
      [](double x) { return 1/x; }),
    [](double x) { cout << x << "\n"; }
  );
}
\end{lstlisting}
\end{block}
\end{frame}

\begin{frame}[t,fragile]{Pipelines por partes}
\begin{block}{x -> x*x -> 1/x -> print}
\begin{lstlisting}
template <typename Execution>
void run_pipe(const Execution & ex, int n)
{
  auto generator = [i=0,max=n] () mutable -> optional<int> {
    if (i<max) return i; else return {};
  };
  auto inner = grppi::pipeline(
      [](int x) -> double { return x*x; },
      [](double x) { return 1/x; });
  auto printer = [](double x) { cout << x << "\n"; };

  grppi::pipeline(ex, generator, inner, printer);
}
\end{lstlisting}
\end{block}
\end{frame}
